
\chapter{Apéndice}
\label{sec:appendix}

{
\color{gray}
\Blindtext[1][1]
}

\section{Capítulo 2}
\label{sec:appendix:sec2}

\paragraph{Función Biyectiva}

Una función es biyectiva es aquella que es a la vez inyectiva y suprayectiva. Es decir, una función entre los elementos de dos conjuntos, donde cada elemento de un conjunto se empareja con exactamente un elemento del otro conjunto, y cada elemento del otro conjunto se empareja con exactamente un elemento del primer conjunto.

Formalmente, dada una función $f$

$ {\begin{array}{rccl}f:&X&\longrightarrow &Y\\&x&\longmapsto &y=f(x)\end{array}} $

Es biyectiva si para todo $y$ de $Y$ existe un único $x$ de $X$ al que la función evaluada en $x$ es igual a $y$

$ \forall y\in Y\;:\quad \exists !\ x\in X\;/\quad f(x)=y $

\paragraph{Problema NP}



\section{Sección 2}
\label{sec:appendix:sec2}


{
\color{gray}
\Blindtext[1][1]

\begin{table}[h]
	\begin{tabularx}{\textwidth}{X | X | X}
		%\hline
		Alpha		& Beta			& Gamma			\\ \hline
		0			& 1				& 2				\\ \hline
		3			& 4				& 5				\\ %\hline
	\end{tabularx}
	\label{tab:table2}
	\caption{This is a caption text.}
\end{table}


\Blindtext[1][2]
}