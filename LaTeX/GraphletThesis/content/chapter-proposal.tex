% !TEX root = ../my-thesis.tex
%
\chapter{Propuesta}
\label{sec:system}

\cleanchapterquote{Innovation distinguishes between a leader and a follower.}{Steve Jobs}{(CEO Apple Inc.)}


{
\color{gray}
\Blindtext[2][1]
}

\section{Graphlets, Órbitas y Roles Sociales}
\label{sec:system:sec1}

\subsection{Graphlets}
\label{sec:intro:results:refs:method}
{
Los Graphlets son subgrafos inducidos pequeños en una red. En teoría de grafos un subgrafo inducido es un subgrafo de otro grafo que se conforma a partir de un subconjunto de vértices y de todas las aristas incidentes a pares de vértices del subconjunto. 
}

\subsection{Orbitas}
\label{sec:intro:results:refs:method}
{
Dentro de los Graphlets existen grupos de simetría de nodos llamados órbitas de automorfismos \cite{sarajlic_graphlet-based_2016} estas describen los distintos roles topológicos en los que un nodo puede participar dentro del Graphlet.
}

\subsection{Relevancia de las Órbitas y su relación con los roles estructurales}
\label{sec:intro:results:refs:method}
{
\color{gray}
\Blindtext[1][2]
}

\section{Perfilar Usuarios}
\label{sec:system:sec2}

{
\color{gray}
\Blindtext[1][2]
}

\begin{figure}[htb]
	\includegraphics[width=\textwidth]{images/Clean-Thesis-Figure}
	\caption{Another Figure example: \textit{(a)} example part one, \textit{(c)} example part two; \textit{(c)} example part three}
	\label{fig:system:example2}
\end{figure}

{
\color{gray}
\Blindtext[2][2]
}

\section{Clusterizar Redes}
\label{sec:system:sec3}
{
\color{gray}
\Blindtext[4][2]
}